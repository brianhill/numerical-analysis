% This paper has been transcribed in Plain TeX by
% David R. Wilkins
% School of Mathematics, Trinity College, Dublin 2, Ireland
% (dwilkins@maths.tcd.ie)
%
% Trinity College, 2002.

\magnification=\magstep1
\vsize=227 true mm \hsize=170 true mm
   \voffset=-0.4 true mm \hoffset=-5.4 true mm

\def\folio{\ifnum\pageno>0 \number\pageno \else
   \ifnum\pageno<0 \romannumeral-\pageno \else\fi\fi}


\font\Largebf=cmbx10  scaled \magstep2
\font\largerm=cmr12
\font\largeit=cmti12
\font\tensc=cmcsc10
\font\sevensc=cmcsc10 scaled 700
\newfam\scfam \def\sc{\fam\scfam\tensc}
\textfont\scfam=\tensc \scriptfont\scfam=\sevensc
\scriptscriptfont\scfam=\sevensc
\font\largesc=cmcsc10 scaled \magstep1
\font\ninerm=cmr9
\newfam\srfam \def\sr{\fam\srfam\ninerm}
\textfont\srfam=\ninerm \scriptfont\srfam=\sevenrm
\scriptscriptfont\srfam=\fiverm

\input epsf

\pageno=0

\null\vskip72pt

\centerline{\Largebf THE MATHEMATICAL PRINCIPLES OF}

\vskip12pt

\centerline{\Largebf NATURAL PHILOSOPHY}

\vskip24pt

\centerline{\Largebf (BOOK 1, SECTION 1)}

\vskip24pt

\centerline{\Largebf By}

\vskip24pt

\centerline{\Largebf Isaac Newton}

\vskip24pt

\centerline{\Largebf Translated into English by}

\vskip24pt

\centerline{\Largebf Andrew Motte}

\vskip36pt

\vfill

\centerline{\largerm Edited by David R. Wilkins}

\vskip 12pt

\centerline{\largerm 2002}

\vskip36pt\eject

\pageno=-1

\null\vskip36pt

\centerline{\Largebf NOTE ON THE TEXT}

\bigskip

Section I in Book I of Isaac Newton's
{\it Philosophi{\ae} Naturalis Principia Mathematica\/}
is reproduced here, translated into English by Andrew Motte.
Motte's translation of Newton's {\it Principia}, entitled
{\it The Mathematical Principles of Natural Philosophy\/}
was first published in 1729.

\bigbreak\bigskip

\line{\hfil David R. Wilkins}

\vskip3pt

\line{\hfil Dublin, June 2002}

\vfill\eject

\pageno=1

\null\vskip36pt

\centerline{\largerm SECTION I.}

\vskip 12pt

\centerline{\largeit Of the method of first and last ratio's of quantities,}

\vskip3pt

\centerline{\largeit by the help whereof we demonstrate the propositions}

\vskip3pt

\centerline{\largeit that follow.}

\nobreak\bigskip

\centerline{\largesc Lemma I.}

\nobreak\bigskip

{\it
Quantities, and the ratio's of quantities, which in any finite
time converge continually to equality, and before the end of that
time approach nearer the one to the other than by any given
difference, become ultimately equal.}

\bigbreak

If you deny it; suppose them to be ultimately unequal, and let
${\rm D}$ be their ultimate difference.  Therefore they cannot
approach nearer to equality than by that given difference
${\rm D}$; which is against the supposition.

\bigbreak

\centerline{\largesc Lemma II.}

\nobreak\bigskip

{\it
If in any figure ${\rm A} \, {\rm a} \, {\rm c} \, {\rm E}$
terminated by the right lines ${\rm A} \, {\rm a}$,
${\rm A} \, {\rm E}$, and the curve
${\rm a} \, {\rm c} \, {\rm E}$, there be inscrib'd any number of
parallelograms ${\rm A} \, {\rm b}$, ${\rm B} \, {\rm c}$,
${\rm C} \, {\rm d}$, \&c.\ comprehended under equal bases
${\rm A} \, {\rm B}$, ${\rm B} \, {\rm C}$, ${\rm C} \, {\rm D}$,
\&c.\ and the sides
${\rm B} \, {\rm b}$, ${\rm C} \, {\rm c}$, ${\rm D} \, {\rm d}$,
\&c.\ parallel to one side ${\rm A} \, {\rm a}$ of the figure;
and the parallelograms
${\rm a} \, {\rm K} \, {\rm b} \, {\rm l}$, ${\rm b} \,
   {\rm L} \, {\rm c} \, {\rm m}$,
${\rm c} \, {\rm M} \, {\rm d} \, {\rm n}$, \&c.\ are compleated.
Then if the breadth of those parallelograms be suppos'd to be
diminished, and their number to be augmented\/ {\rm in
infinitum:} I say that the ultimate ratio's which the inscrib'd
figure
${\rm A} \, {\rm K} \, {\rm b} \, {\rm L} \, {\rm c} \, {\rm M} \,
   {\rm d} \, {\rm D}$,
the circumscribed figure
${\rm A} \, {\rm a} \, {\rm l} \, {\rm b} \, {\rm m} \, {\rm c} \,
   {\rm n} \, {\rm d} \, {\rm o} \, {\rm E}$,
and the curvilinear figure
${\rm A} \, {\rm a} \, {\rm b} \, {\rm c} \, {\rm d} \, {\rm E}$,
will have to one another, are ratio's of equality.}

\midinsert
\centerline{\epsfbox{b1s1_fg.1}}
\endinsert

\bigbreak

For the difference of the inscrib'd and circumscrib'd figures is
the sum of the parallelograms $K \, l$, $L \, m$, $M \, n$,
$D \, o$, that is, (from the equality of all their bases) the
rectangle under one of their bases $K \, b$ and the sum of their
altitudes $A \, a$, that is, the rectangle $A \, B \, l \, a$.
But this rectangle, because its breadth $A \, B$ is suppos'd
diminished {\it in infinitum}, becomes less than any given space.
And therefore (By Lem.~I.) the figures inscribed and
circumscribed become ultimately equal one to the other; and much
more will the intermediate curvilinear figure be ultimately equal
to either.
{\it Q.E.D.}

\bigbreak

\centerline{\largesc Lemma III.}

\nobreak\bigskip

{\it
The same ultimate ratio's are also ratio's of equality, when the
breadths, ${\rm A} \, {\rm B}$, ${\rm B} \, {\rm C}$,
${\rm D} \, {\rm C}$, \&c.\ of the parallelograms are unequal,
and are all diminished\/ {\rm in infinitum.}}

\bigbreak

For suppose $A \, F$ equal to the greatest breadth, and compleat
the parallelogram ${\rm F} \, {\rm A} \, {\rm a} \, {\rm f}$.
This parallelogram will be greater than the difference of the
inscrib'd and circumscribed figures; but, because its breadth
$A \, F$ is diminished {\it in infinitum}, it will become less
than any given rectangle.
{\it Q.E.D.}

{\sc Cor.}~1.
Hence the ultimate sum of those evanescent parallelograms will in
all parts coincide with the curvilinear figure.

{\sc Cor.}~2.
Much more will the rectilinear figure, comprehended under the
chords of the evanescent arcs $a \, b$, $b \, c$, $c \, d$,
\&c.\ ultimately coincide with the curvilinear figure.

{\sc Cor.}~3.
And also the circumscrib'd rectilinear figure comprehended under
the tangents of the same arcs.

{\sc Cor.}~4.
And therefore these ultimate figures (as to their perimeters $a
\, c \, E$,) are not rectilinear, but curvilinear limits of
rectilinear figures.

\bigbreak

\centerline{\largesc Lemma IV.}

\nobreak\bigskip

{\it
If in two figures
${\rm A} \, {\rm a} \, {\rm c} \, {\rm E}$,
${\rm P} \, {\rm p} \, {\rm r} \, {\rm T}$
you inscribe (as before) two ranks of parallelograms, an equal
number in each rank, and when their breadths are diminished\/
{\rm in infinitum,} the ultimate ratio's of the parallelograms in
one figure to those in the other, each to each respectively, are
the same; I say that those two figures
${\rm A} \, {\rm a} \, {\rm c} \, {\rm E}$,
${\rm P} \, {\rm p} \, {\rm r} \, {\rm T}$,
are to one another in that same ratio.}

\midinsert
\centerline{\epsfbox{b1s1_fg.2}}
\endinsert

\bigbreak

For as the parallelograms in the one are severally to the
parallelograms in the other, so (by composition) is the sum of
all in the one to the sum of all in the other; and so is the one
figure to the other; because (by Lem.~3.) the former figure to
the former sum, and the latter figure to the latter sum are both
in the ratio of equality.
{\it Q.E.D.}

{\sc Cor.}
Hence if two quantities of any kind are any how divided into an
equal number of parts: and those parts, when their number is
augmented and their magnitude diminished {\it in infinitum}, have
a given ratio one to the other, the first to the first, the
second to the second, and so on in order: the whole quantities
will be one to the other in that same given ratio.  For if, in
the figures of this lemma, the parallelograms are taken one to
the other in the ratio of the parts, the sum of the parts will
always be as the sum of the parallelograms; and therefore
supposing the number of the parallelograms and parts to be
augmented, and their magnitudes diminished {\it in infinitum},
those sums will be in the ultimate ratio of the parallelogram in
the one figure to the correspondent parallelogram in the other;
that is, (by the supposition) in the ultimate ratio of any part
of the one quantity to the correspondent part of the other.

\bigbreak

\centerline{\largesc Lemma V.}

\nobreak\bigskip

{\it
In similar figures, all sorts of homologous sides, whether
curvilinear or rectilinear, are proportional; and the area's are
in the duplicate ratio of the homologous sides.}

\bigbreak

\centerline{\largesc Lemma VI.}

\nobreak\bigskip

{\it
If any arc ${\rm A} \, {\rm C} \, {\rm B}$ given in position is
subtended by its chord ${\rm A} \, {\rm B}$, and in any point
${\rm A}$ in the middle of the continued curvature, is touch'd by
a right line ${\rm A} \, {\rm D}$, produced both ways; then if
the points ${\rm A}$ and ${\rm B}$ approach one another and meet,
I say the angle ${\rm B} \, {\rm A} \, {\rm D}$, contained
between the chord and the tangent, will be diminished
{\it in infinitum}, and ultimately will vanish.}

\midinsert
\centerline{\epsfbox{b1s1_fg.3}}
\endinsert

\bigbreak

For if that angle does not vanish, the arc $A \, C \, B$ will
contain with the tangent $A \, D$ an angle equal to a rectilinear
angle; and therefore the curvature at the point~${\rm A}$ will
not be continued, which is against the supposition.

\bigbreak

\centerline{\largesc Lemma VII.}

\nobreak\bigskip

{\it
The same things being supposed; I say, that the ultimate ratio of
the arc, chord, and tangent, any one to any other, is the ratio
of equality.}

\bigbreak

For while the point~${\rm B}$ approaches towards the
point~${\rm A}$, consider always $A \, B$ and $A \, D$ as
produc'd to the remote points $b$ and $d$, and
parallel to the secant $B \, D$ draw $b \, d$: and let the arc
$A \, c \, b$ be always similar to the arc $A \, C \, B$.  Then
supposing the points ${\rm A}$ and ${\rm B}$ to coincide, the
angle $d \, A \, b$ will vanish, by the preceding lemma; and
therefore the right lines $A \, b$, $A \, d$ (which are always
finite) and the intermediate arc $A \, c \, b$ will coincide, and
become equal among themselves.  Wherefore the right lines
$A \, B$, $A \, D$, and the intermediate arc $A \, C \, B$ (which
are always proportional to the former) will vanish; and
ultimately acquire the ratio of equality.
{\it Q.E.D.}

\midinsert
\centerline{\epsfbox{b1s1_fg.4}}
\endinsert

{\sc Cor.}~1.
Whence if through ${\rm B}$ we draw $B \, F$ parallel to the
tangent, always cutting any right line $A \, F$ passing through
${\rm A}$ in ${\rm F}$; this line $B \, F$ will be ultimately in
the ratio of equality with the evanescent arc $A \, C \, B$;
because, compleating the parallelogram $A \, F \, B \, D$, it is
always in a ratio of equality with $A \, D$.

{\sc Cor.}~2.
And if through ${\rm B}$ and ${\rm A}$ more right lines are drawn
as $B \, E$, $B \, D$, $A \, F$, $A \, G$ cutting the tangent
$A \, D$ and its parallel $B \, F$; the ultimate ratio of all the
abscissa's $A \, D$, $A \, E$, $B \, F$, $B \, G$, and of the
chord and arc $A \, B$, any one to any other, will be the ratio
of equality.

{\sc Cor.}~3.
And therefore in all our reasoning about ultimate ratio's, we may
freely use any one of those lines for any other.

\bigbreak

\centerline{\largesc Lemma VIII.}

\nobreak\bigskip

{\it
If the right lines ${\rm A} \, {\rm R}$, ${\rm B} \, {\rm R}$,
with the arc ${\rm A} \, {\rm C} \, {\rm B}$, the chord
${\rm A} \, {\rm B}$, and the tangent ${\rm A} \, {\rm D}$,
constitute three triangles
${\rm R} \, {\rm A} \, {\rm B}$,
${\rm R} \, {\rm A} \, {\rm C} \, {\rm B}$,
${\rm R} \, {\rm A} \, {\rm D}$,
and the points ${\rm A}$ and ${\rm B}$ approach and meet: I say
that the ultimate form of these evanescent triangles is that of
similitude, and their ultimate ratio that of equality.}

\midinsert
\centerline{\epsfbox{b1s1_fg.3}}
\endinsert

\bigbreak

For while the point~${\rm B}$ approaches towards the point
${\rm A}$ consider always $A \, B$, $A \, D$, $A \, R$, as
produced to the remote points $b$, $d$, and $r$,
and $r \, b \, d$ as drawn parallel to $R \, D$, and
let the arc $A \, c \, b$ be always similar to the arc
$A \, C \, B$.  Then supposing the points ${\rm A}$ and ${\rm B}$
to coincide, the angle $b \, A \, d$ will vanish; and therefore
the three triangles
$r \, A \, b$, $r \, A \, c \, b$, $r \, A \, d$,
(which are always finite) will coincide, and on that account
become both similar and equal.  And therefore the triangles
$R \, A \, B$, $R \, A \, C \, B$, $R \, A \, D$, which are
always similar and proportional to these.  will ultimately become
both similar and equal among themselves.
{\it Q.E.D.}

{\sc Cor.}
And hence in all our reasonings about ultimate ratio's, we may
indifferently use any one of those triangles for any other.

\bigbreak

\centerline{\largesc Lemma IX.}

\nobreak\bigskip

{\it
If a right line ${\rm A} \, {\rm E}$, and a curve line
${\rm A} \, {\rm B} \, {\rm C}$,
both given by position, cut each other in a given angle ${\rm
A}$; and to that right line, in another given angle,
${\rm B} \, {\rm D}$, ${\rm C} \, {\rm E}$,
are ordinately applied, meeting the curve in
${\rm B}$,~${\rm C}$; and the points ${\rm B}$ and ${\rm C}$
together, approach towards, and meet in, the point ${\rm A}$: I
say that the area's of the triangles
${\rm A} \, {\rm B} \, {\rm D}$,
${\rm A} \, {\rm C} \, {\rm E}$,
will ultimately be one to the other in the duplicate ratio of the
sides.}

\midinsert
\centerline{\epsfbox{b1s1_fg.5}}
\endinsert

\bigbreak

For while the points $B$, $C$ approach towards the
point $A$, suppose always $A \, D$ to be produced to the
remote points $d$ and $e$, so as $A \, d$, $A \, e$
may be proportional to $A \, D$, $A \, E$; and the ordinates
$d \, b$, $e \, c$, to be drawn parallel to the ordinates
$D \, B$ and $E \, C$, and meeting $A \, B$ and $A \, C$ produced
in $b$ and $c$.  Let the curve $A \, b \, c$ be
similar to the curve $A \, B \, C$, and draw the right line
$A \, g$ so as to touch both curves in $A$, and cut the
ordinates $D \, B$, $E \, C$, $d \, b$, $e \, c$, in $F$,
$G$, $f$, $g$.  Then supposing the length
$A \, e$ to remain the same, let the points $B$ and
$C$ meet in the point~$A$; and the angle
$c \, A \, g$ vanishing, the curvilinear areas $A \, b \, d$,
$A \, c \, e$ will coincide with the rectilinear areas
$A \, f \, d$, $A \, g \, e$; and therefore (by Lem.~5) will be
one to the other in the duplicate ratio of the sides $A \, d$,
$A \, e$.  But the areas $A \, B \, D$, $A \, C \, E$ are always
proportional to these areas, and so the sides $A \, D$, $A \, E$
are to these sides.  And therefore the areas
$A \, B \, D$, $A \, C \, E$ are ultimately one to the other in
the duplicate ratio of the sides $A \, D$, $A \, E$.
{\it Q.E.D.}

\bigbreak

\centerline{\largesc Lemma X.}

\nobreak\bigskip

{\it
The spaces which a body describes by any finite force urging it,
whether that force is determined and immutable, or is continually
augmented or continually diminished, are in the very beginning
of the motion one to the other in the duplicate ratio of the
times.}

\bigbreak

Let the times be represented by the lines $A \, D$, $A \, E$, and
the velocities generated in those times by the ordinates
$D \, B$, $E \, C$.  The spaces described with these velocities
will be as the areas $A \, B \, D$, $A \, C \, E$, described by
those ordinates, that is, at the very beginning of the motion (by
Lem.~9) in the duplicate ratio of the times $A \, D$, $A \, E$.
{\it Q.E.D.}

{\sc Cor.}~1.
And hence one may easily infer, that the errors of bodies
describing similar parts of similar figures in proportional
times, are nearly in the duplicate ratio of the times in which
they are generated; if so be these errors are generated by any
equal forces similarly applied to the bodies, and measur'd by the
distances of the bodies from those places of the similar figures,
at which, without the action of those forces, the bodies would
have arrived in those proportional times.

{\sc Cor.}~2.
But the errors that are generated by proportional forces
similarly applied to the bodies at similar parts of the similar
figures, are as the forces and the squares of the times
conjunctly.

{\sc Cor.}~3.
The same thing is to be understood of any spaces whatsoever
described by bodies urged with different forces.  All which, in
the very beginning of the motion, are as the forces and the
squares of the times conjunctly.

{\sc Cor.}~4.
And therefore the forces are as the spaces described described in
the very beginning of the motion directly, and the squares of the
times inversly.

{\sc Cor.}~5.
And the squares of the times are as the spaces describ'd directly
and the forces inversly.

\bigbreak

\centerline{\largesc Scholium.}

\nobreak\bigskip

If in comparing indetermined quantities of different sorts one
with another, any one is said to be as any other directly or
inversly: the meaning is, that the former is augmented or
diminished in the same ratio with the latter, or with its
reciprocal.  And if any one is said to be as any other two or
more directly or inversly: the meaning is, that the first is
augmented or diminished in the ratio compounded of the ratio's in
which the others, or the reciprocals of the others, are augmented
or diminished.  As if ${\rm A}$ is said to be as ${\rm B}$
directly and ${\rm C}$ directly and ${\rm D}$ inversly: the
meaning is, that ${\rm A}$ is augmented or diminished in the same
ratio with $\displaystyle B \times C \times {1 \over D}$, that is
to say, that ${\rm A}$ and
$\displaystyle {B \, C \over D}$
are one to the other in a given ratio.

\bigbreak

\centerline{\largesc Lemma XI.}

\nobreak\bigskip

{\it
The evanescent subtense of the angle of contact, in all curves,
which at the point of contact have a finite curvature, is
ultimately in the duplicate ratio of the subtense of the
conterminate arc.}

\midinsert
\centerline{\epsfbox{b1s1_fg.6}}
\endinsert

\bigbreak

{\sc Case}~1.
Let $A \, B$ be that arc, $A \, D$ its tangent, $B \, D$ the
subtense of the angle of contact perpendicular on the tangent,
$A \, B$ the subtense of the arc.  Draw $B \, G$ perpendicular to
the subtense $A \, B$, and $A \, G$ to the tangent $A \, D$,
meeting in $G$; then let the points $D$, $B$
and $G$, approach to the points $d$, $b$ and
$g$, and suppose $J$ to be the ultimate intersection
of the lines $B \, G$, $A \, G$, when the points $D$,
$B$ have come to $A$.  It is evident that the
distance $G \, J$ may be less than any assignable.  But (from the
nature of the circles passing through the points $A$,~$B$,~$G$;
$A$,~$b$,~$g$)
$A \, B^2 = A \, G \times B \, D$,
and
$A \, b^2 = A \, g \times b \, d$;
and therefore the ratio of $A \, B^2$ to $A \, b^2$ is compounded
of the ratio's of $A \, G$ to $A \, g$ and of $B \, D$ to
$b \, d$.  But because $G \, J$ may be assum'd of less length
than any assignable, the ratio of $A \, G$ to $A \, g$ may be
such as to differ from the ratio of equality by less than any
assignable difference; and therefore the ratio of $A \, B^2$ to
$A \, b^2$ may be such as to differ from the ratio of $B \, D$ to
$b \, d$ by less than any assignable difference.  Therefore, by
Lem.~1. the ultimate ratio of $A \, B^2$ to $A \, b^2$ is the
same with the ultimate ratio of $B \, D$ to $b \, d$.
{\it Q.E.D.}

{\sc Case}~2.
Now let $B \, D$ be inclined to $A \, D$ in any given angle, and
the ultimate ratio of $B \, D$ to $b \, d$ will always be the
same as before, and therefore the same with the ratio of
$A \, B^2$ to $A \, b^2$.
{\it Q.E.D.}

{\sc Case}~3.
And if we suppose the angle~$D$ not to be given, but that
the right line $B \, D$ converges to a given point, or is
determined by any other condition whatever; nevertheless, the
angles $D$, $d$, being determined by the same law,
will always draw nearer to equality, and approach nearer to each
other than by any assigned difference, and therefore, by Lem.~1,
will at last be equal, and therefore the lines $B \, D$, $b \, d$
are in the same ratio to each other as before.
{\it Q.E.D.}

{\sc Cor.}~1.
Therefore since the tangents $A \, D$, $A \, d$, the arcs
$A \, B$, $A \, b$, and their sines $B \, C$, $b \, c$, become
ultimately equal to the chords $A \, B$, $A \, b$; their squares
will ultimately become as the subtenses $B \, D$, $b \, d$.

{\sc Cor.}~2.
Their squares are also ultimately as the versed sines of the
arcs, bisecting the chords, and converging to a given point.  For
those versed sines are as the subtenses $B \, D$, $b \, d$.

{\sc Cor.}~3.
And therefore the versed sine is in the duplicate ratio of the
time in which a body will describe an arc with a given velocity.

{\sc Cor.}~4.
The rectilinear triangles $A \, D \, B$, $A \, d \, b$ are
ultimately in the triplicate ratio of the sides $A \, D$,
$A \, d$, and in a sesquiplicate ratio of the sides $D \, B$, $d
\, b$; as being in the ratio compounded of the sides $A \, D$ to
$D \, B$, and of $A \, d$ to $d \, b$.  So also the triangles
$A \, B \, C$, $A \, b \, c$ are ultimately in the triplicate
ratio of the sides $B \, C$, $b \, c$.  What I call the
sesquiplicate ratio is the subduplicate of the triplicate, as
been compounded of the simple and subduplicate ratio.

{\sc Cor.}~5.
And because $D \, B$, $d \, b$ are ultimately parallel and in the
duplicate ratio of the lines $A \, D$, $A \, d$: the ultimate
curvilinear areas $A \, D \, B$, $A \, d \, b$ will be (by the
nature of the parabola) two thirds of the rectilinear triangles
$A \, D \, B$, $A \, d \, b$; and the segments $A \, B$, $A \, b$
will be one third of the same triangles.  And thence those areas
and those segments will be in the triplicate ratio as well of the
tangents $A \, D$, $A \, d$; as of the chords and arcs $A \, B$,
$A \, b$.

\bigbreak

\centerline{\largesc Scholium.}

\nobreak\bigskip

But we have all along supposed the angle of contact to be neither
infinitely greater nor infinitely less, than the angles of
contact made by circles and their tangents; that is, that the
curvature at the point~${\rm A}$ is neither infinitely small nor
infinitely great, or that the interval $A \, J$ is of a finite
magnitude.  For $D \, B$ may be taken as $A \, D^3$: in which
case no circle can be drawn through the point ${\rm A}$, between
the tangent $A \, D$ and the curve $A \, B$, and therefore the
angle of contact will be infinitely less than those of circles.
And by a like reasoning if $D \, B$ be made successively as
$A \, D^4$, $A \, D^5$, $A \, D^6$, $A \, D^7$, \&c.\ we shall
have a series of angles of contact, proceeding {\it in
infinitum}, wherein every succeeding term is infinitely less than
the preceding.  And if $D \, B$ be made successively as
$A \, D^2$, $A \, D^{3 \over 2}$, $A \, D^{4 \over 3}$,
$A \, D^{5 \over 4}$, $A \, D^{6 \over 5}$, $A \, D^{7 \over 6}$,
\&c.\ we shall have another infinite series of angles of contact,
the first of which is of the same sort with those of circles, the
second infinitely greater, and every succeeding one infinitely
greater than the preceding.  But between any two of these angles
another series of intermediate angles of contact may be
interposed proceeding both ways {\it in infinitum}, wherein every
succeeding angle shall be infinitely greater, or infinitely less
than the preceding.  As if between the terms $A \, D^2$ and
$A \, D^3$ there were interposed the series
$A \, D^{13 \over 6}$,
$A \, D^{11 \over 5}$, $A \, D^{9 \over 4}$,
$A \, D^{7 \over 3}$, $A \, D^{5 \over 2}$,
$A \, D^{8 \over 3}$, $A \, D^{11 \over 4}$,
$A \, D^{14 \over 5}$, $A \, D^{17 \over 6}$, \&c.
And again between any two angles of this series, a new series of
intermediate angles may be interpolated, differing from one
another by infinite intervals.  Nor is nature confin'd to any
bounds.

Those things which have been demonstrated of curve lines and the
superficies which they comprehend, may be easily applied to the
curve superficies and contents of solids.  These lemmas are
premised, to avoid the tediousness of deducing perplexed
demonstrations {\it ad absurdum}, according to the method of the
ancient geometers.  For demonstrations are more contracted by the
method of indivisibles: But because the hypothesis of
indivisibles seems somewhat harsh, and therefore that method is
reckoned less geometrical; I chose rather to reduce the
demonstrations of the following propositions to the first and
last sums and ratio's of nascent and evanescent quantities, that
is, to the limits of those sums and ratio's; and so to premise,
as short as I could, the demonstrations of those limits.  For
hereby the same thing is perform'd as by the method of
indivisibles; and now those principles being demonstrated, we may
use them with more safety.  Therefore if hereafter, I should
happen to consider quantities as made up of particles, or should
use little curve lines for right ones; I would not be understood
to mean indivisibles, but evanescent divisible quantities; not
the sums and ratio's of determinate parts, but always the limits
of sums and ratio's: and that the force of such demonstrations
always depends on the method lay'd down in the foregoing lemma's.

Perhaps it may be objected, that there is no ultimate proportion
of evanescent quantities; because the proportion, before the
quantities have vanished, is not the ultimate, and when they are
vanished, is none.  But by the same argument it may be alledged,
that a body arriving at a certain place, and there stopping, has
no ultimate velocity; because the velocity, before the body comes
to the place, is not its ultimate velocity; when it has arrived,
is none.  But the answer is easy; for by the ultimate velocity is
meant that with which the body is moved, neither before it
arrives at its last place and the motion ceases, nor after, but
at the very instant it arrives; that is, that velocity with which
the body arrives at its last place, and with which the motion
ceases.  An in like manner, by the ultimate ratio of evanescent
quantities is to be understood the ratio of the quantities, not
before they vanish, nor afterwards, but with which they vanish.
In like manner the first ratio of nascent quantities is that with
which they begin to be.  And the first or last sum is that with
which they begin and cease to be (or to be augmented or
diminished.)  There is a limit which the velocity at the end of
the motion may attain, but not exceed.  This is the ultimate
velocity.  And there is the like limit in all quantities and
proportions that begin and cease to be.  And since such limits
are certain and definite, to determine the same is a problem
strictly geometrical.  But whatever is geometrical we may be
allowed to use in determining and demonstrating any other thing
that is likewise geometrical.

It may also be objected, that if the ultimate ratio's of
evanescent quantities are given, their ultimate magnitudes will
be also given: and so all quantities will consist of
indivisibles, which is contrary to what {\it Euclid\/} has
demonstrated concerning incommensurables, in the 10th book of his
Elements.  But this objection is founded on a false supposition.
For those ultimate ratio's with which quantities vanish, are not
truly the ratio's of ultimate quantities, but limits towards
which the ratio's of quantities, decreasing without limit, do
always converge; and to which they approach nearer than by any
given difference, but never go beyond, nor in effect attain to,
till the quantities are diminished {\it in infinitum}.  This
thing will appear more evident in quantities infinitely great.
If two quantities, whose difference is given, be augmented
{\it in infinitum}, the ultimate ratio of these quantities will
be given, to wit, the ratio of equality; but it does not from
thence follow, that the ultimate or greatest quantities
themselves, whose ratio that is, will be given.  Therefore if in
what follows, for the sake of being more easily understood, I
should happen to mention quantities as least, or evanescent, or
ultimate; you are not to suppose that the quantities of any
determinate magnitude are meant, but such as are conceiv'd to be
always diminished without end.

\bye

