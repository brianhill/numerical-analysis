% Numerical Analysis on a Pocket Calculator
% Day 2

\def\folio{\ifnum\pageno>0 \number\pageno \else
   \ifnum\pageno<0 \romannumeral-\pageno \else\fi\fi}

\font\largebf=cmbx10  scaled \magstep2
\font\largerm=cmr12
\font\largeit=cmti12
\font\tensc=cmcsc10
\font\sevensc=cmcsc10 scaled 700
\newfam\scfam \def\sc{\fam\scfam\tensc}
\textfont\scfam=\tensc \scriptfont\scfam=\sevensc
\scriptscriptfont\scfam=\sevensc
\font\largesc=cmcsc10 scaled \magstep1
\font\ninerm=cmr9
\newfam\srfam \def\sr{\fam\srfam\ninerm}
\textfont\srfam=\ninerm \scriptfont\srfam=\sevenrm
\scriptscriptfont\srfam=\fiverm

\input epsf

\pageno=3

\null\vskip36pt

\centerline{\largerm DAY 2}
\nobreak\bigskip

\centerline{\largeit Getting Started with the Hewlett Packard HP-25}
\nobreak\bigskip

\noindent{\bf Downloading an Emulation}
\nobreak\bigskip

\noindent On iPhone:
\nobreak\bigskip

\noindent GO-25 SciRPN by Stephen Lidie and RPN-25 CE  by CuVee Software are both apps I have tried. At present, I am not aware of any reason to choose one or the other. They are only a few dollars each. Get both and experiment with which you like better?
\bigskip

\noindent On Android:
\nobreak\bigskip
\noindent Brandon and Kuba have found and installed go25c by Olivier De Smet. It is free.
\bigskip

\noindent{\bf Getting Started}
\nobreak\bigskip

\noindent You have Sections 1-3 of the {\it Hewlett-Packard HP-25 Owner's Handbook.} Work through Section 1 carefully, with the calculator emulation in hand, and do all of the suggested exercises. If your answers don't agree with those in the {\it Owner's Handbook,} resolve any discrepancies.
\bigskip

\noindent {\bf To Share}
\nobreak\bigskip

\noindent Once you have worked through Section 1, you should already be able to do simple calculations. By ``simple,'' I mean calculations that only involve addition, subtraction, multiplication, division, squares, square roots, and powers.

Choose any simple calculation that you have needed to do recently. If you can't remember one, make one up. Do the calculation on the HP-25 emulation. Write out the keystrokes needed to do the calculation so that you are prepared to share it with others and to hand in.

\bigskip

\noindent {\bf Other Preparation and Plans}
\nobreak\bigskip

\noindent I am willing to do more trig review. In particular, we have not yet discussed the utility of radians. A bonus derivation from Max will be a geometrical proof of the equation,

$$\sin(\alpha + \beta) = \sin \alpha \cos \beta + \cos \alpha \sin \beta.$$

\noindent Then I am hoping we have enough time to work through most of Sections 2 and 3 of the {\it Owner's Handbook} together.

\noindent 
\bye
