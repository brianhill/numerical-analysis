% Numerical Analysis on a Pocket Calculator
% Day 2

\def\folio{\ifnum\pageno>0 \number\pageno \else
   \ifnum\pageno<0 \romannumeral-\pageno \else\fi\fi}

\font\largebf=cmbx10  scaled \magstep2
\font\largerm=cmr12
\font\largeit=cmti12
\font\tensc=cmcsc10
\font\sevensc=cmcsc10 scaled 700
\newfam\scfam \def\sc{\fam\scfam\tensc}
\textfont\scfam=\tensc \scriptfont\scfam=\sevensc
\scriptscriptfont\scfam=\sevensc
\font\largesc=cmcsc10 scaled \magstep1
\font\ninerm=cmr9
\newfam\srfam \def\sr{\fam\srfam\ninerm}
\textfont\srfam=\ninerm \scriptfont\srfam=\sevenrm
\scriptscriptfont\srfam=\fiverm

\input epsf

\pageno=4

\null\vskip36pt

\centerline{\largerm DAY 3}
\nobreak\bigskip

\centerline{\largeit Calculating with the HP-25 --- Includes the First Problem Set to Turn in}
\nobreak\bigskip

\noindent{\bf Continue Reading the HP-25 Owner's Handbook}
\nobreak\bigskip

\noindent Sections 4 and 5 are what remains of the {\it Owner's Handbook} (apart from front matter and appendices). They are across from the copier. We are going to need a few classes to get through this.

Continue working through the {\it Owner's Handbook,} through most of Section~4: specifically, work all the way up to and including p.~59 (where trigonometric functions are introduced). The calculator
by default works in degrees, and we will work in degrees. If you switch to radians, remember to switch it back.

The remainder of this document is a Problem Set to turn in.

\bigskip

\noindent{\bf Problem 1 --  The Cosine Addition Law}
\nobreak\bigskip

\noindent Max gave a proof of the sine addition law. Use his methods to give a proof of the cosine addition law. Copy this figure carefully (perhaps at a larger size) onto your problem set solution.
Without even considering the dashed line, you can find three right triangles in this figure.

\midinsert
\centerline{\epsfbox{na_day03_fig.1}}
\endinsert
\nobreak\bigskip


\noindent Find the right triangle in this figure that has angle $\alpha + \beta$ and hypotenuse 1. Using this triangle, label what you can using the CAH in SOH-CAH-TOA. You are off to the races! Now it is a matter of
carefully finding and labeling other angles and dimensions that are known. This is really quite a puzzle, but it is also completely doable with perseverance.

\bigskip

\noindent{\bf Problem 2 --  Area of a Triangle}
\nobreak\bigskip


\noindent Recall this formula from Day 1 (which took at least 10 minutes for me to derive!),

$$A={1\over 2} b^2 {\sin\theta \sin\phi \over \sin (\theta + \phi)},$$

\noindent which gives the area of this triangle:

\midinsert
\centerline{\epsfbox{na_day03_fig.2}}
\endinsert
\nobreak\bigskip

\noindent With $\theta = 30^{\circ},$ $\phi = 45^{\circ},$ and $b = 4\thinspace{\rm cm},$ what does your HP-25 give you for the area? (Include units.)

\bigskip

\noindent{\bf Problem 3 -- Keystrokes}
\nobreak\bigskip

\noindent Repeat the computation in Problem 2. Record---using the same style that is being used in the {\rm Owner's Handbook}---the keystrokes you used to get your answer.

\bigskip

\noindent{\bf Problems 4 and 5 -- Your Inventions}
\nobreak\bigskip

\noindent Having read through much of Section~4, you now have powers, scientific notation, squares, square roots, percentages, reciprocals, $\pi,$ and trig functions at your disposal. If you look a little
further ahead, you will find the logarithmic and exponential functions, both in base 10 and in base $e$.

Your assignment for Problems 4 and 5 is to choose two formulas that you have encountered, explain what the formulas represent, 
give numbers to insert for the variables, and report the expected answers. In addition to the original you hand in, come to class with six more copies of your Problems 4 and 5 to share. We will aim to work through at least one of every person's example problems.

\bye
