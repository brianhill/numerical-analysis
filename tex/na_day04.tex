% Numerical Analysis on a Pocket Calculator
% Day 4

\def\folio{\ifnum\pageno>0 \number\pageno \else
   \ifnum\pageno<0 \romannumeral-\pageno \else\fi\fi}

\font\largebf=cmbx10  scaled \magstep2
\font\largerm=cmr12
\font\largeit=cmti12
\font\tensc=cmcsc10
\font\sevensc=cmcsc10 scaled 700
\newfam\scfam \def\sc{\fam\scfam\tensc}
\textfont\scfam=\tensc \scriptfont\scfam=\sevensc
\scriptscriptfont\scfam=\sevensc
\font\largesc=cmcsc10 scaled \magstep1
\font\ninerm=cmr9
\newfam\srfam \def\sr{\fam\srfam\ninerm}
\textfont\srfam=\ninerm \scriptfont\srfam=\sevenrm
\scriptscriptfont\srfam=\fiverm

\input epsf

\pageno=6

\null\vskip36pt

\centerline{\largerm DAY 4}
\nobreak\bigskip

\centerline{\largeit The Stack, Registers, and Programming}
\nobreak\bigskip

\noindent{\bf Where are we?}
\nobreak\bigskip

\noindent You have read to and worked all the examples up to p.~59 in the {\it Owner's Handbook}. For now, we are going to skip most of the rest of Section~4. What we are skipping is HMS conversion, which is some stuff for converting things like 12h~15m~30s to 12.25833333h and back, polar coordinate conversion, and finally, some handy and sophisticated stuff for statistics using the $\Sigma+$ and $\Sigma-$ keys. As we are doing applications later in the course, we will get back to those things.

However, some things buried in the pages we are skipping that we should learn are logarithms, exponentials, and powers. The calculator has base~10 logarithms and exponentials, it has natural logarithms and exponentials (base~$e$), and if that isn't enough it will raise any base to any power using the $y^x$~key. So read pp.~63-65 starting with ``Logarithmic and Exponential Functions'' and finishing with the horribly complicated formula for Mach number on p.~65. Key the whole thing in and make sure you get the expected answer.

The reason we are skipping most of pp.~60-71 is that we want to get to programming!

\bigskip

\noindent{\bf The Stack and Registers}
\nobreak\bigskip

\noindent We have the stack  (X, Y, Z, and T), Last X, and REG 0, REG 1, REG 2, ..., REG 7 at our disposal. We have these both when we are doing calculations manually and when we are programming the calculator. It is extremely common, when running a program, to have to enter initial values into one or more of these locations before starting the program.

\bigskip

\noindent{\bf Our First Program}
\nobreak\bigskip

\noindent We will do the program alluded to in the reading: the area of a sphere. The formula is $A = 4 \pi r^2$.
\bigskip

\noindent{\bf Your Second Program}
\nobreak\bigskip

\noindent To test yourself that you know how to do another simple program, do a program for $V = {4\over 3} \pi r^3$.

\bigskip

\noindent{\bf Your Third Program: Nimb}
\nobreak\bigskip

\noindent Key in and play the attached program. People in the combinatorics class call this ``Baby Nim.'' We just call it Nimb. Find somebody in the combinatorics class and show it to them.

\bye
