% Numerical Analysis on a Pocket Calculator
% Day 7

\def\folio{\ifnum\pageno>0 \number\pageno \else
   \ifnum\pageno<0 \romannumeral-\pageno \else\fi\fi}

\font\largebf=cmbx10  scaled \magstep2
\font\largerm=cmr12
\font\largeit=cmti12
\font\tensc=cmcsc10
\font\sevensc=cmcsc10 scaled 700
\newfam\scfam \def\sc{\fam\scfam\tensc}
\textfont\scfam=\tensc \scriptfont\scfam=\sevensc
\scriptscriptfont\scfam=\sevensc
\font\largesc=cmcsc10 scaled \magstep1
\font\ninerm=cmr9
\newfam\srfam \def\sr{\fam\srfam\ninerm}
\textfont\srfam=\ninerm \scriptfont\srfam=\sevenrm
\scriptscriptfont\srfam=\fiverm

\input epsf

\pageno=18

\null\vskip36pt

\centerline{\largerm DAY 7}
\nobreak\bigskip

\centerline{\largeit Applications: Graphing and Base Conversion --- Includes Problem Set 3 Due Sept.~20 }
\nobreak\bigskip

\noindent{\bf Where are We?}
\nobreak\bigskip

\noindent We have finished (with the exception of only a few topics --- notably the statistical functions) the entire {\it Owner's Handbook}, we have keyed in several programs, including, most recently, Moon Lander, and we have written a program that finds the direction of Mecca. The HP-25 {\it Applications Programs} book is what we will draw from next for a rich variety of subjects and programs. The first chapter of applications is on Algebra and Number Theory.

\bigskip

\noindent{\bf Graphing}
\nobreak\bigskip

\noindent Read pp.~4--8 of the {\it Applications Programs} book. Then do Problems 1 to 3 on the Problem Set. These will take a while. Budget extra time to get Problem~2 right. You will have to significantly modify the graphing program and document your new program.

\bigskip

\noindent{\bf Base Conversion}
\nobreak\bigskip

\noindent Take a look at pp.~22--25 of the {\it Applications Programs} book. Then do Problems 4 and 5 on the Problem Set. These will be pretty quick, except for keying in the programs. Leave the last program (the one on pp.~24-25) keyed in when we start the next class so that we can try to understand how it works.

\bigskip

\noindent{\bf History}
\nobreak\bigskip

\noindent We have pp.~180-189 from {\it Bill \& Dave} to read. It is fun, but straightforward. I'm not expecting it to provoke a lot of discussion. Perhaps I am wrong. Come with your comments.

\bigskip

\noindent{\bf Problem Set 3}
\nobreak\bigskip

\noindent The problems are on the following 3 pages.

\bigskip

\noindent{\bf Looking Ahead}
\nobreak\bigskip

\noindent If anyone has studied complex numbers and would like to discuss some of the other programs that we are skipping, come see me.

The material on cross products, dot products, and simultaneous equations in two unknowns is all worthy and not too advanced, so we will cover that next.

\bye
