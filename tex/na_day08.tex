% Numerical Analysis on a Pocket Calculator
% Day 8

\def\folio{\ifnum\pageno>0 \number\pageno \else
   \ifnum\pageno<0 \romannumeral-\pageno \else\fi\fi}

\font\largebf=cmbx10  scaled \magstep2
\font\largerm=cmr12
\font\largeit=cmti12
\font\tensc=cmcsc10
\font\sevensc=cmcsc10 scaled 700
\newfam\scfam \def\sc{\fam\scfam\tensc}
\textfont\scfam=\tensc \scriptfont\scfam=\sevensc
\scriptscriptfont\scfam=\sevensc
\font\largesc=cmcsc10 scaled \magstep1
\font\ninerm=cmr9
\newfam\srfam \def\sr{\fam\srfam\ninerm}
\textfont\srfam=\ninerm \scriptfont\srfam=\sevenrm
\scriptscriptfont\srfam=\fiverm

\input epsf

\pageno=24

\null\vskip36pt

\centerline{\largerm DAY 8}
\nobreak\bigskip

\centerline{\largeit Applications: Dot Products and Simultaneous Equations}
\nobreak\bigskip

\noindent{\bf Where are We?}
\nobreak\bigskip

\noindent In class on Day 7, we had enough time to go through Dot Products.

\bigskip

\noindent{\bf Simultaneous Equations}
\nobreak\bigskip

\noindent Prepare for Day 8 by reading the introductory material on simultaneous equations, keying in the associated program, and doing an example.

\bigskip

\noindent{\bf HP History --- Basics}
\nobreak\bigskip

\noindent We have pp.~264-282 from {\it Bill \& Dave} to read. Your previous reading covered the development of the HP-9100A released in 1968. This reading covers the development of the HP-35, released in 1972. We are trying to understand the origin of the HP-25 released in 1975 (or the HP-25C, released in 1976). There is a critical stepping stone between the HP-35 and the HP-25. It is the HP-65 released in 1974. So that this doesn't just become a blur of product releases, try to keep in mind what differentiated the HP-9100A, the HP-35, the HP-65, and the HP-25 from each other. It is somewhat analogous to the generations of the iPod and iPhone (the iPod, the iPhone, the iPhone 3G, and the iPhone 3GS, introduced 2001, 2007, 2008, and 2009 respectively).

\bigskip

\noindent{\bf HP History --- Big Questions}
\nobreak\bigskip

\noindent The reading closes by saying that HP now outsells Apple, and therefore HP missing the boat on the Apple computer that the Woz designed while he was working for them is not that big a deal. However, the book was written in 2007, when Apple was still struggling and the first iPhone was just about to come out. Today, the market capitalization of Apple is \$2.5 trillion and HP is \$26 billion. In other words, Apple is almost 100x times as valuable a company as HP, and HP is now a truly sad shadow of its former self. The Carly Fiorina era at HP was the last straw, just as the John Sculley, Michael Spindler, and Gil Amelio eras at Apple almost finished Apple off.

My questions for you are why do all big companies decline (severely) a few decades (or at most a century) after their founding? Are there counter-examples? Most importantly is there evidence of the problems emerging at HP in this reading? If you have a different and better opening question to discuss in the wake of the reading, feel free to bring it.

\bigskip

\noindent{\bf Looking Ahead}
\nobreak\bigskip

\noindent Your next problem set will involve dot products, simultaneous equations, and finance applications. That problem set will be your fourth and it will be in the Day 9 handout.

\bigskip

\bye
