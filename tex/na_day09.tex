% Numerical Analysis on a Pocket Calculator
% Day 8

\def\folio{\ifnum\pageno>0 \number\pageno \else
   \ifnum\pageno<0 \romannumeral-\pageno \else\fi\fi}

\font\largebf=cmbx10  scaled \magstep2
\font\largerm=cmr12
\font\largeit=cmti12
\font\tensc=cmcsc10
\font\sevensc=cmcsc10 scaled 700
\newfam\scfam \def\sc{\fam\scfam\tensc}
\textfont\scfam=\tensc \scriptfont\scfam=\sevensc
\scriptscriptfont\scfam=\sevensc
\font\largesc=cmcsc10 scaled \magstep1
\font\ninerm=cmr9
\newfam\srfam \def\sr{\fam\srfam\ninerm}
\textfont\srfam=\ninerm \scriptfont\srfam=\sevenrm
\scriptscriptfont\srfam=\fiverm

\input epsf

\pageno=24

\null\vskip36pt

\centerline{\largerm DAY 9 -- Includes 4th Problem Set Due Sep. 27}
\nobreak\bigskip

\centerline{\largeit Applications: Finance}
\nobreak\bigskip

\noindent{\bf Where are We?}
\nobreak\bigskip

\noindent Today we are luanching into the second chapter in the HP-25 {\it Applications Programs} book.

\bigskip

\noindent{\bf Finance}
\nobreak\bigskip

\noindent Most of us will (unfortunately) have to take out a loan at some point in our lives. If you are financially well-off, perhaps you will get to collect interest on a loan that you make to someone else. Either way, it is important to know how loan balances are calculated. On pp.~32--33 the {\it Applications Programs} book describes the standard way to do the calculation. Read those two pages carefully.

\bigskip

\bigskip

\noindent{\bf Looking Ahead}
\nobreak\bigskip

\noindent Your next problem set will involve dot products, simultaneous equations, and finance applications. That problem set will be your fourth and it will be in the Day 9 handout.

\bigskip

\bye
