% Numerical Analysis on a Pocket Calculator
% Day 9

\def\folio{\ifnum\pageno>0 \number\pageno \else
   \ifnum\pageno<0 \romannumeral-\pageno \else\fi\fi}

\font\largebf=cmbx10  scaled \magstep2
\font\largerm=cmr12
\font\largeit=cmti12
\font\tensc=cmcsc10
\font\sevensc=cmcsc10 scaled 700
\newfam\scfam \def\sc{\fam\scfam\tensc}
\textfont\scfam=\tensc \scriptfont\scfam=\sevensc
\scriptscriptfont\scfam=\sevensc
\font\largesc=cmcsc10 scaled \magstep1
\font\ninerm=cmr9
\newfam\srfam \def\sr{\fam\srfam\ninerm}
\textfont\srfam=\ninerm \scriptfont\srfam=\sevenrm
\scriptscriptfont\srfam=\fiverm

\input epsf

\pageno=26

\null\vskip36pt

\centerline{\largerm DAY 9 -- Includes 4th Problem Set Due Sep. 27}
\nobreak\bigskip

\centerline{\largeit Applications: Finance}
\nobreak\bigskip

\noindent{\bf Where are We?}
\nobreak\bigskip

\noindent  We are launching into the second chapter in the HP-25 {\it Applications Programs} book. We should have had a good discussion of the theory on pp.~32-33 in the Day~8 class. The amount of money paid in interest on a 30-year loan (relative to the principal amount) is instructive because it can be shockingly large.

\bigskip

\noindent{\bf Assignment 4---Preamble}
\nobreak\bigskip

\noindent For the first problem I wanted to create a realistic and instructive use of the first program in the chapter on finance.

For the second problem, I wanted to create an interesting and instructive programming assignment that is not too hard and not too easy, and isn't already done in the book. If you have a better problem that you would like to do and present instead, come talk to me!

\bigskip

\noindent{\bf Assignment 4---Problem 1---Interest Payments}

\bigskip

\noindent Key in the program on pp.~34-35 and update the example to use modern values. Use end of October 2021 as the time the first payment is made. Make the loan amount \$500,300. Make the interest rate 6\%. Make the payment amount \$3000. This loan will be paid off in 30 years.

What was the accumulated interest for 2021 (periods 1--3)? What will be the accumulated interest for periods 4--15? What balance remained at the end of 2021? What will remain at the end of 2022? Generate a table showing the schedule of interest paid and remaining balance for the first 5 years of the mortgage (periods 12, 24, 36, 48, and 60).

\bigskip

\noindent{\bf Assignment 4---Problem 2---Portfolio Value}
\nobreak\bigskip

\noindent A ``portfolio'' is a collection of stocks (and bonds) that you have invested in. You try to choose companies whose stocks are going up because you expect their profits to go up. Money is not just made when the stock price goes up. It is made each quarter when and if the company has profits and declares a dividend. The value of a bond can actually decline (how, why!) even if it is not at risk of default, so they behave somewhat like stocks even though the amount you are paid for holding a bond is fixed. The portfolio program you write will just have two stocks.

Your program will compute the value of a portfolio containing two stocks. You have some number of shares of the first stock. Let's make that stock Apple (ticker symbol AAPL not APPL) and let's say you bought 100 shares on Jan. 1, 2021 when the stock was trading at 132.00.  Let's make the second stock Coca-Cola (ticker symbol KO) and say you bought 200 shares of KO on Jan. 1, 2021 when the stock was trading at 55.00. So your initial portfolio value is \$24,200.

The quarterly dividend history for AAPL for the 7 quarters since Jan. 1 2021 has been \$0.205, \$0.205, \$0.22, \$0.22, \$0.22, \$0.23, \$0.23, \$0.23. The quarterly dividend history for KO for the 7 quarters since Jan. 1 2021 has been 0.42, 0.42, 0.42, 0.42, 0.44, 0.44, and 0.44. You can find the stock price history easily by googling stock price KO.

Assume that the dividends received from the AAPL stock are reinvested in AAPL stock. Assume that the dividends from the KO stock are reinvested in the KO stock.

Write a program that stops to get the AAPL stock price, the AAPL dividend amount, and stops again to get the KO stock price, and the KO dividend amount. After those have been entered, have your program stop one more time to report the portfolio value. {\it Problem 3 is on the reverse.}

\noindent{\bf Assignment 4---Problem 3---Portfolio Graph}

\bigskip

\noindent Using the program you have just written, make a table and a graph showing the initial portfolio value and the portfolio value following each of the quarter ends. Quarter ends are at March 31, June 30, September 30, and December 31 each year. Since we haven't quite gotten to September 30, 2022 just use the current stock price of AAPL and KO as the final stock prices. As of this writing (September 23, 2022), those were \$150.01 for AAPL and \$58.58 for KO. Your table should have all the input values you used for quarterly dividends and stock prices as well as the program's result for the portfolio value for each quarter.

\bigskip

\bye
