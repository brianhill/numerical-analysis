% Numerical Analysis on a Pocket Calculator
% Day 12

\def\folio{\ifnum\pageno>0 \number\pageno \else
   \ifnum\pageno<0 \romannumeral-\pageno \else\fi\fi}

\font\largebf=cmbx10  scaled \magstep2
\font\largerm=cmr12
\font\largeit=cmti12
\font\tensc=cmcsc10
\font\sevensc=cmcsc10 scaled 700
\newfam\scfam \def\sc{\fam\scfam\tensc}
\textfont\scfam=\tensc \scriptfont\scfam=\sevensc
\scriptscriptfont\scfam=\sevensc
\font\largesc=cmcsc10 scaled \magstep1
\font\ninerm=cmr9
\newfam\srfam \def\sr{\fam\srfam\ninerm}
\textfont\srfam=\ninerm \scriptfont\srfam=\sevenrm
\scriptscriptfont\srfam=\fiverm

\input epsf

\pageno=31

\null\vskip36pt

\centerline{\largerm DAY 12}
\nobreak\bigskip

\centerline{\largeit Midterm Exam}
\nobreak\bigskip

\noindent {\it During this exam, you can use your copy of the {\rm HP-25 Owner's Handbook} and the {\rm HP-25 Applications Programs} book, and your notes and problem sets. However, I deliberately made an exam that uses all the ideas we have been developing, but applies them to an {\it entirely} new subject.}

\bigskip

\noindent {\it The tables you need to make are already started on an attached sheet.}

\bigskip

\noindent {\it Half the exam is Problem 4, which is a programming problem. There are three program forms attached so that you have some extra chances to correct errors and re-copy your program.}

\bigskip

\noindent{\bf Problem 1 {\it (3 pts)}}
\nobreak\bigskip

\noindent Assume there are $R$ rabbits and $W$ wolves. For Problem~1 let $W=0$ (no wolves). If nothing eats the rabbits, they would just keep multiplying. Let's assume that the relevant time period for assessing multiplication is one month and that if you have $R$ rabbits at the beginning of a month, then you have $(1 + r)R$ rabbits at the end of the month. In the real world, you must have an integer number of rabbits, but we won't let that interfere with our model-making.

Make a table that has $R=30$ rabbits at the beginning, and a multiplication rate $r=0.5$. So at Month~0 in your table, you will have 30~rabbits, and at Month~1, you will have 45~rabbits, and at Month~2, you will have 67.5~rabbits. Fill in your table using the multiplying rule all the way to Month~4.

\bigskip

\noindent{\bf Problem 2  {\it (2 pts)}}
\nobreak\bigskip

\noindent For Problem 2, assume there are no rabbits ($R=0$). Assume there are 6 wolves initially $W=6$. In our model, wolves starve if there are no rabbits to eat. We will call the starvation rate $w$. The formula is if you have $W$ wolves at the beginning of the month, then you have $(1-w)W$ wolves at the end of the month. Let's have half the wolves starve each month if there are no rabbits, so $w=0.5$. So at Month~0 in your wolves table, you will have 6~wolves, at Month~1, you will have 3~wolves, and at Month~2, you will have 1.5~wolves. Fill in your table all the way to Month~4.

\bigskip

\noindent {\it Now things are going to get really interesting!! We are going to have both rabbits and wolves. Wolves eat rabbits and when they do, they can multiply instead of starving.}

\bigskip

\noindent{\bf Problem 3  {\it (5 pts)}}

\bigskip

\noindent Here is the new model. It is called the predatory-prey model. In this model, if there are $W$ wolves and $R$ rabbits at the beginning of the month, then at the end of the month there are:

$$(1 + r)R - e R W$$

\noindent rabbits, and at the end of the month, there are:

$$(1 - w)W + f R W$$

\noindent wolves. Let's start with $R=30$ rabbits and $W=6$ wolves. So $R W = 180$ initially. Let's have $e = 0.1$ and $f = 0.025$.

Use these formulas to make a table for just for Months 0 to 3. It will be helpful in the table to have $R W$ as a column because you need that for both formulas.  In Month 0, you will have $R=30, W=6,$ and $R W=180$. In Month~1, you will have $R=27$ and $W=7.5$.

\vfil\break

\noindent{\bf Problem 4  {\it (10 pts)}}
\bigskip\nobreak
\noindent Write a program that does Problem 3.

\bigskip

\noindent Do it however you like, but if you don't have an outline in mind immediately, I suggest the following:

\bigskip

\noindent Have the user put $R$ in REG0, $W$ in REG1, $r$ in REG2, $w$ in REG3, $e$ in REG4, and $f$ in REG5.

\bigskip

\noindent Outline:

\bigskip

\noindent 1.~Calculate $RW$ from the old $R$ and $W$ and update REG6.

\noindent 2.~Stop and display $RW$.

\noindent 3.~Calculate the new $R$ and update REG0.

\noindent 4.~Stop and display $R$.

\noindent 5.~Calculate the new $W$ and update REG1.

\noindent 6.~Stop and display $W$.

\noindent 7.~Go back to Step 1.

\bigskip

\noindent Use your program to quickly and efficiently fill in the table on the next page out to Month~6.

\bigskip

\noindent All done? Bored? Fill in a table for this problem that goes out to Month 18. The wolves and the rabbits will recover from their disastrously low population levels.

\bye
